\documentclass[a4paper,11pt, oneside]{book}
\usepackage[utf8]{inputenc}
\usepackage[francais]{babel}
\usepackage[T1]{fontenc}
\usepackage{graphicx}
\usepackage{float}
\usepackage{wrapfig}
\usepackage{setspace}
\usepackage{geometry}
\usepackage{multicol}
\usepackage{etoolbox}
\usepackage{color}
\usepackage[explicit,pagestyles]{titlesec}
\usepackage[absolute,overlay]{textpos}
\usepackage{fancyhdr}
\usepackage{fontspec}
\usepackage{eurosym}
\usepackage{titlesec}


% ====== CONFIG ========
\setmainfont{Roboto Light}
\setsansfont{Roboto}
\setmonofont{Roboto}
\newfontfamily\light{Roboto Slab Light}
\graphicspath{{img/}}
\setlength{\unitlength}{1mm}

\makeatletter

\definecolor{primary}{RGB}{44, 62, 80}


\titleformat{\chapter}[display]{\huge}{\thechapter \quad #1}{0pt}{}
\titleformat{\section}[display]{\LARGE}{}{0pt}{\thesection \quad #1}


\setlength{\TPHorizModule}{1mm}
\setlength{\TPVertModule}{1mm}
\def\sizeMedia{38}
\def\size{3.8cm}
\def\sizeMargin{0.2cm}
\def\margin{2}
\def\fixMargin{0}

\pagestyle{plain}


\title{Compte rendu d'activité}
\author{Yann Prono}
\date{\today}

\def\school{TELECOM Nancy}
\def\schoolAddress{193 Avenue Paul Muller}
\def\schoolPostalCode{54602}
\def\schoolCity{Villers-lès-Nancy}
\def\schoolCodeAndCity{\schoolPostalCode, \schoolCity}
\def\schoolYear{2016 - 2017}

\def\appName{The Octopus Challenge}

\def\club{Club Studio}
\def\chair{Eliot GODARD}
\def\secretary{Yann PRONO}
\def\banker{Ansel GAMET}
\def\teacher{Isabelle HEUDIARD}
\def\bde{Victor CHOLLEY-BARROYER}

\def\schoolYear{2016 - 2017}

% ====== END CONFIG ========


\begin{document}

	\begin{titlepage}
		\thispagestyle{empty}

{\color{primary}
	\begin{flushleft}
		\school\\
		\schoolAddress\\
		\schoolCodeAndCity\\
		Université de Lorraine\\
	\end{flushleft}

	\vspace{0.6cm}

	\begin{center}
		\rule{\textwidth}{0.8pt}

			\vspace{0.5cm}
			\baselineskip=3pt
			{\Huge \bfseries{Compte rendu d'activité}}\\
			\vspace{0.2cm}
			{\huge \bfseries{\club}}
			\vspace{0.5cm}

		\rule{\textwidth}{0.8pt}
	\end{center}

	\vspace{0.2cm}

	\begin{center}
		\includegraphics[width=0.8\textwidth]{logo_studio.png}
	\end{center}

	\begin{textblock}{210}(0, 230)
			\color{white}

			\begin{center}
				{\LARGE{\schoolYear}}
			\end{center}

			\vspace{0.1cm}
			\begin{center}
			{\large
				\begin{tabular}{lcr}
					Président & \hspace{1cm} & \chair \\
					Secrétaire & & \secretary\\
					Trésorier & & \banker\\
					& & \\
					Professeur référent & & \teacher\\
					Référent BDE & & \bde\\
				\end{tabular}
			}
			\end{center}
			\color{black}
	\end{textblock}

	\begin{textblock}{210}(0, 168)
	\begin{center}
		\includegraphics[width=3.5cm, keepaspectratio]{cover.png}
	\end{center}
	\end{textblock}
}

	\end{titlepage}


	\newpage

	\newpage\null\thispagestyle{empty}\newpage
	\tableofcontents

	\chapter{Introduction}
	Lorem ipsum dolor sit amet, consectetur adipisicing elit, sed do eiusmod tempor incididunt ut labore et dolore magna aliqua. Ut enim ad minim veniam, quis nostrud exercitation ullamco laboris nisi ut aliquip ex ea commodo consequat. Duis aute irure dolor in reprehenderit in voluptate velit esse cillum dolore eu fugiat nulla pariatur. Excepteur sint occaecat cupidatat non proident, sunt in culpa qui officia deserunt mollit anim id est laborum.
	\setcounter{page}{1}
	\clearpage

	\chapter{\appName}
	\clearpage

	\chapter{User Documentation}
	Welcome to Octopus challenge.
	Octopus-challenge was a quiz web site about the English culture.\\
	Now we will present you how to use Octopus Challenge.
	for the next of the presentation of the application, we use this plan:
	\begin{itemize}
		\item Navigation menu
		\item Home page
		\item How to play the game
		\item Best score
		\item Send a question request\\
	\end{itemize}

	\section{Navigation menu}
	The navigation menu was placed on the top of each page of the web site and this offer for the player 5 Button to navigate into the different screen of the web site.
	\begin{itemize}
		\item ‘the Game’: This Button redirect you to the home page of the application.
		\item ‘play now!’: This Button redirect you to the game page of the application.
		\item ‘Best scores’: This Button redirect you to the score page where you can see the best score perform on the application.
		\item ‘suggests a question’: This Button redirects you to a page where you can suggest a question for the game.
		\item Logo Button : This button has the same action as the Button ‘The game’ but has a special design randomly set when you load the page.
	\end{itemize}
	
	\section{Home page}
	When the user log on Octopus challenges he arrives on the Home page screen. 
(figure 1)

//Home Screen figure 1

The Home screen has various sections.
\begin{itemize}
	\item The ‘Start game’ section. This section allows the user to start a game
	\item The ‘Information’ section. This section presents some information about the game rules and some story for the game.
\end{itemize}
	
	\chapter{Description technique}

	Ce chapitre présentera en détails les contraintes de ce projet, la manière dont il a été développé ainsi que les problèmes que nous avons dues faire face
	durant les semaines de développement.

	\section{Contraintes}

	Nous avons dégagé deux types de contraintes lors de la phase d'analyse de ce projet:
	\begin{itemize}
		\item Les contraintes liés au sujet (temps de développement, attentes du client).
		\item Les contraintes liés au type de l'outil informatique à développer.\\
	\end{itemize}

Nous avons mis beaucoup de temps à choisir l'outil informatique que nous souhaitions développer. Notre premier choix s'est porté sur un clone du projet Voltaire adapté à la langue de shakespeare.
Cependant, ce choix ne s'est pas révélé motivant. Le choix suivant s'est alors porté sur un quiz à base d'image où l'objectif devait être de trouver le mot correspondant au mot affiché. De plus, ce choix
ne nous semblait pas assez solide pour en faire un outil informatique à part entière. De plus, Ce moyen d'apprendre la langue n'est pas forcément efficace.
Notre choix final s'est porté sur un mélange de tous les choix précédents en y apportant quelque chose d'éducatif et simple à utiliser.\\

Le choix du sujet nous a fait perdre du temps pour la conception et le développement de l'outil. Le développement devait donc se faire sur une plus courte période.
Le développement d'une application web s'est alors présenté à nous. Nous disposons tout les deux des compétences pour travailler dans le web.
De plus, Le web est accessible depuis n'importe quel appareil disposant d'un navigateur internet, point non négligeable pour le succès de l'application.\\

Le sujet étant fixé, d'autres contraintes se sont rajoutés. La première est le fait que l'application de doit être accessible depuis n'importe quel appareil. En effet,
une application ne fonctionne pas et ne s'affiche pas de la même manière sur un ordinateur ou sur un smartphone (écran plus petit, limitations matérielles...). Il sera donc nécéssaire
d'adapter certains aspects de l'application pour la rendre fonctionnelle. La dernière contraintes est l'expérience de l'utilisateur. Si notre application à pour ambition d'améliorer
la compréhension de l'Anglais, la manière dont les informations sont transmises de l'application à l'utilisateur est un point crucial. L'information doit être facilement accessible (peu de clicks, accès rapide, application réactive).
Il saura donc important d'adapter l'application afin dé répondre à cette contrainte dans l'objectif d'apporter la meilleur expérience pour l'utilisateur.


\section{Développement de l'application}


L'application \appName est composé deux parties:
\begin{itemize}
	\item le backend, programme informatique étant exécuté sur le serveur, ayant pour but de répondre aux requêtes des clients.
	\item le frontend, programme étant exécuté sur le navigateur web, coeur du projet, permettant de gérer l'affichage des informations sur le navigateur.\\
\end{itemize}

\noindent Pour ce projet, nous avons décidé de développer le backend avec express, un framework NodeJS permettant de développer rapidement un serveur web. Le serveur permet
\begin{itemize}
	\item D'envoyer le point d'entrée de l'application web avec les programmes du frontend.
	\item De stocker l'ensemble des questions grâce à une base de données.
	\item De stocker l'ensemble des scores des joueurs grâce à une base de données.
	\item d'envoyer l'ensemble des questions disponibles via une API JSON.\\
\end{itemize}

\noindent Pour la base de données, nous avons optés pour PostgreSQL. Ce choix n'a pas d'importance réelle pour l'application mais s'est révelé efficace pour l'hébergement de l'application.
Le serveur est la petite partie de ce projet (environ 300 lignes de code). Le coeur du projet concerne le frontend de l'application (3000 lignes de code).\\

Le frontend a été développé avec VueJS 2. La manière de concevoir des applications web a beaucoup évolué au cours de ces dernières années avec notamment de nombreux frameworks comme AngularS, ReactJS ou encore EmbedJS. VueJS fait partie
de ces frameworks rendant les applications plus maintenables et facile à développer. Une fonctionnalité intéressante TODO\\

\noindent Le frontend permet:
\begin{itemize}
		\item D'afficher la page correspondante à la requête de l'utilisateur (page d'accueil, jouer, pages des scores ...)
		\item De mettre en oeuvre le moteur de jeu développé pourle quiz (affichage de la question, affichage des réponses, lecture d'une vidéo youtube...)
		\item D'envoyer les résultats d'une partie au serveur
		\item D'afficher le formulaire pour la suggestion d'une question
		\item
\end{itemize}
\section{Hébergement}

Une des problématiques dont nous avons dues faire face à la fin du développement de l'application est l'hébergement. Cette étape finale du développement à pour but de
rendre disponible l'application sur internet. De nombreux hébergeurs proposent ce service gratuitement ou non. Cependant, la plupart ne se sont pas tous mis à niveau concernant
les nouvelles technologies comme NodeJS. C'est pourquoi, il a donc été difficile de trouver un hébérgeur supportant la technologie comme nous avons employés.\\

Un hébergeur a retenu notre attention: Heroku. La principale force d'héroku est qu'il répond au problème actuel précédent, le support des nouvelles technologies pour le web.
Un des autres points forts est que l'hébergement est gratuit pour de petites applications comme la nôtre. Il a été nécéssaire de configurer l'application et l'hérgement sur Heroku (base de données, port d'écoute ...)
afin de rendre fonctionnelle l'application sur internet. Une contrainte nous a cependant été imposée: Utiliser postgresSQL comme outil pour la base de données. Vous retrouverez l'application à l'adresse suivante:

\section{Difficultées rencontrées}

Nous avons dues faire face à deux difficultées au cours de ce projet. La première a été la technologie utilisée pour le développment du frontend. Nous n'avions jamais utilisé ce framework.
Nous avons donc appris comment utiliser VueJS à l'aide de tutoriaux ainsi que des vidéos. Cette difficultée a été cependant fructueuse car elle nous permis d'agrandir notre bagage de compétences
mais également à apprendre une technologie en une courte période.

\chapter{Conclusion}


\end{document}
