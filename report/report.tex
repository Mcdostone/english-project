\documentclass[a4paper,11pt, oneside]{book}
\usepackage[utf8]{inputenc}
\usepackage[francais]{babel}
\usepackage[T1]{fontenc}
\usepackage{graphicx}
\usepackage{float}
\usepackage{wrapfig}
\usepackage{setspace}
\usepackage{geometry}
\usepackage{multicol}
\usepackage{etoolbox}
\usepackage{color}
\usepackage[explicit,pagestyles]{titlesec}
\usepackage[absolute,overlay]{textpos}
\usepackage{fancyhdr}
\usepackage{fontspec}
\usepackage{eurosym}
\usepackage{titlesec}


% ====== CONFIG ========
\setmainfont{Roboto Light}
\setsansfont{Roboto}
\setmonofont{Roboto}
\newfontfamily\light{Roboto Slab Light}
\graphicspath{{img/}}
\setlength{\unitlength}{1mm}

\makeatletter

\definecolor{primary}{RGB}{44, 62, 80}


\titleformat{\chapter}[display]{\huge}{\thechapter \quad #1}{0pt}{}
\titleformat{\section}[display]{\LARGE}{}{0pt}{\thesection \quad #1}


\setlength{\TPHorizModule}{1mm}
\setlength{\TPVertModule}{1mm}
\def\sizeMedia{38}
\def\size{3.8cm}
\def\sizeMargin{0.2cm}
\def\margin{2}
\def\fixMargin{0}

\pagestyle{plain}


\title{Compte rendu d'activité}
\author{Yann Prono}
\date{\today}

\def\school{TELECOM Nancy}
\def\schoolAddress{193 Avenue Paul Muller}
\def\schoolPostalCode{54602}
\def\schoolCity{Villers-lès-Nancy}
\def\schoolCodeAndCity{\schoolPostalCode, \schoolCity}
\def\schoolYear{2016 - 2017}

\def\appName{The Octopus Challenge}

\def\club{Club Studio}
\def\chair{Eliot GODARD}
\def\secretary{Yann PRONO}
\def\banker{Ansel GAMET}
\def\teacher{Isabelle HEUDIARD}
\def\bde{Victor CHOLLEY-BARROYER}

\def\schoolYear{2016 - 2017}

% ====== END CONFIG ========


\begin{document}

	\begin{titlepage}
		\thispagestyle{empty}

{\color{primary}
	\begin{flushleft}
		\school\\
		\schoolAddress\\
		\schoolCodeAndCity\\
		Université de Lorraine\\
	\end{flushleft}

	\vspace{0.6cm}

	\begin{center}
		\rule{\textwidth}{0.8pt}

			\vspace{0.5cm}
			\baselineskip=3pt
			{\Huge \bfseries{Compte rendu d'activité}}\\
			\vspace{0.2cm}
			{\huge \bfseries{\club}}
			\vspace{0.5cm}

		\rule{\textwidth}{0.8pt}
	\end{center}

	\vspace{0.2cm}

	\begin{center}
		\includegraphics[width=0.8\textwidth]{logo_studio.png}
	\end{center}

	\begin{textblock}{210}(0, 230)
			\color{white}

			\begin{center}
				{\LARGE{\schoolYear}}
			\end{center}

			\vspace{0.1cm}
			\begin{center}
			{\large
				\begin{tabular}{lcr}
					Président & \hspace{1cm} & \chair \\
					Secrétaire & & \secretary\\
					Trésorier & & \banker\\
					& & \\
					Professeur référent & & \teacher\\
					Référent BDE & & \bde\\
				\end{tabular}
			}
			\end{center}
			\color{black}
	\end{textblock}

	\begin{textblock}{210}(0, 168)
	\begin{center}
		\includegraphics[width=3.5cm, keepaspectratio]{cover.png}
	\end{center}
	\end{textblock}
}

	\end{titlepage}


	\newpage

	\newpage\null\thispagestyle{empty}\newpage
	\setcounter{page}{1}
	\tableofcontents

	\chapter{Introduction}
	Lorem ipsum dolor sit amet, consectetur adipisicing elit, sed do eiusmod tempor incididunt ut labore et dolore magna aliqua. Ut enim ad minim veniam, quis nostrud exercitation ullamco laboris nisi ut aliquip ex ea commodo consequat. Duis aute irure dolor in reprehenderit in voluptate velit esse cillum dolore eu fugiat nulla pariatur. Excepteur sint occaecat cupidatat non proident, sunt in culpa qui officia deserunt mollit anim id est laborum.
	\clearpage

	\chapter{\appName}
	\clearpage

	\chapter{Description technique}

	Ce chapitre présentera en détails les contraintes de ce projet, la manière dont il a été développé ainsi que les problèmes que nous avons dues faire face
	durant les semaines de développement.

	\section{Contraintes}

	Nous avons dégagé deux types de contraintes lors de la phase d'analyse de ce projet:
	\begin{itemize}
		\item Les contraintes liés au sujet (temps de développement, attentes du client).
		\item Les contraintes liés au type de l'outil informatique à développer.\\
	\end{itemize}

Nous avons mis beaucoup de temps à choisir l'outil informatique que nous souhaitions développer. Notre premier choix s'est porté sur un clone du projet Voltaire adapté à la langue de shakespeare.
Cependant, ce choix ne s'est pas révélé motivant. Le choix suivant s'est alors porté sur un quiz à base d'image où l'objectif devait être de trouver le mot correspondant au mot affiché. De plus, ce choix
ne nous semblait pas assez solide pour en faire un outil informatique à part entière. De plus, Ce moyen d'apprendre la langue n'est pas forcément efficace.
Notre choix final s'est porté sur un mélange de tous les choix précédents en y apportant quelque chose d'éducatif et simple à utiliser.\\

Le choix du sujet nous a fait perdre du temps pour la conception et le développement de l'outil. Le développement devait donc se faire sur une plus courte période.
Le développement d'une application web s'est alors présenté à nous. Nous disposons tout les deux des compétences pour travailler dans le web.
De plus, Le web est accessible depuis n'importe quel appareil disposant d'un navigateur internet, point non négligeable pour le succès de l'application.\\

Le sujet étant fixé, d'autres contraintes se sont rajoutés. La première est le fait que l'application de doit être accessible depuis n'importe quel appareil. En effet,
une application ne fonctionne pas et ne s'affiche pas de la même manière sur un ordinateur ou sur un smartphone (écran plus petit, limitations matérielles...). Il sera donc nécéssaire
d'adapter certains aspects de l'application pour la rendre fonctionnelle. La dernière contraintes est l'expérience de l'utilisateur. Si notre application à pour ambition d'améliorer
la compréhension de l'Anglais, la manière dont les informations sont transmises de l'application à l'utilisateur est un point crucial. L'information doit être facilement accessible (peu de clicks, accès rapide, application réactive).
Il saura donc important d'adapter l'application afin dé répondre à cette contrainte dans l'objectif d'apporter la meilleur expérience pour l'utilisateur.


\clearpage
\section{Développement de l'application}


L'application \appName est composé deux parties:
\begin{itemize}
	\item le backend, programme étant exécuté sur le serveur, ayant pour but de répondre aux requètes des clients.
	\item le frontend, programme étant exécuté sur le navigateur web, coeur du projet, permettant de gérer l'affichage des informations sur le navigateur.
\end{itemize}





\end{document}
