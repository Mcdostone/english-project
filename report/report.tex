\documentclass[a4paper,11pt, oneside]{book}
\usepackage[utf8]{inputenc}
\usepackage[francais]{babel}
\usepackage[T1]{fontenc}
\usepackage{graphicx}
\usepackage{float}
\usepackage{wrapfig}
\usepackage{setspace}
\usepackage{geometry}
\usepackage{multicol}
\usepackage{etoolbox}
\usepackage{color}
\usepackage[explicit,pagestyles]{titlesec}
\usepackage[absolute,overlay]{textpos}
\usepackage{fancyhdr}
\usepackage{fontspec}
\usepackage{eurosym}
\usepackage{titlesec}


% ====== CONFIG ========

\setmainfont{Roboto Light}

\setmainfont{Roboto Light}
\setsansfont{Roboto}
\setmonofont{Roboto}
\newfontfamily\light{Roboto Slab Light}
\graphicspath{{img/}}
\setlength{\unitlength}{1mm}

\makeatletter

\definecolor{primary}{RGB}{44, 62, 80}


\titleformat{\chapter}[display]{\huge}{\thechapter \quad #1}{0pt}{}
\titleformat{\section}[display]{\LARGE}{}{0pt}{\thesection \quad #1}


\setlength{\TPHorizModule}{1mm}
\setlength{\TPVertModule}{1mm}
\def\sizeMedia{38}
\def\size{3.8cm}
\def\sizeMargin{0.2cm}
\def\margin{2}
\def\fixMargin{0}

\pagestyle{plain}


\title{Compte rendu d'activité}
\author{Yann Prono}
\date{\today}

\def\school{TELECOM Nancy}
\def\schoolAddress{193 Avenue Paul Muller}
\def\schoolPostalCode{54602}
\def\schoolCity{Villers-lès-Nancy}
\def\schoolCodeAndCity{\schoolPostalCode, \schoolCity}
\def\schoolYear{2016 - 2017}

\def\appName{The Octopus Challenge}
\def\octopusName{Lord Octopus}

\def\club{Club Studio}
\def\chair{Eliot GODARD}
\def\secretary{Yann PRONO}
\def\banker{Ansel GAMET}
\def\teacher{Isabelle HEUDIARD}
\def\bde{Victor CHOLLEY-BARROYER}

\def\schoolYear{2016 - 2017}

% ====== END CONFIG ========


\begin{document}

	\begin{titlepage}
		\thispagestyle{empty}

{\color{primary}
	\begin{flushleft}
		\school\\
		\schoolAddress\\
		\schoolCodeAndCity\\
		Université de Lorraine\\
	\end{flushleft}

	\vspace{0.6cm}

	\begin{center}
		\rule{\textwidth}{0.8pt}

			\vspace{0.5cm}
			\baselineskip=3pt
			{\Huge \bfseries{Compte rendu d'activité}}\\
			\vspace{0.2cm}
			{\huge \bfseries{\club}}
			\vspace{0.5cm}

		\rule{\textwidth}{0.8pt}
	\end{center}

	\vspace{0.2cm}

	\begin{center}
		\includegraphics[width=0.8\textwidth]{logo_studio.png}
	\end{center}

	\begin{textblock}{210}(0, 230)
			\color{white}

			\begin{center}
				{\LARGE{\schoolYear}}
			\end{center}

			\vspace{0.1cm}
			\begin{center}
			{\large
				\begin{tabular}{lcr}
					Président & \hspace{1cm} & \chair \\
					Secrétaire & & \secretary\\
					Trésorier & & \banker\\
					& & \\
					Professeur référent & & \teacher\\
					Référent BDE & & \bde\\
				\end{tabular}
			}
			\end{center}
			\color{black}
	\end{textblock}

	\begin{textblock}{210}(0, 168)
	\begin{center}
		\includegraphics[width=3.5cm, keepaspectratio]{cover.png}
	\end{center}
	\end{textblock}
}

	\end{titlepage}


	\newpage

	\newpage\null\thispagestyle{empty}\newpage
	\setcounter{page}{1}
	\tableofcontents

	\chapter{Introduction}

Is this the fiftheen time you passed the TOEIC test and you still didn't get it ?
Are you bored about making uninteresting tests on abandoned websites?
Do you want to improve your english level by listening dialogs of your preferred TV serie ?
Do you want to challenge others people about your language skills ?
Do you sing songs that you don't understand ?\\

\noindent English is everywhere: music, TV series, movies, articles, documentation, comics strip, social networks, youtube videos etc...
All these supports are gold nuggets. Indeed, a lot of people use these media in their daily lives in their native language.
The interesting point of this lifestyle is the possibility to translate these media in other language in order to learn it.

\appName \ is a quiz developped in the english courses context. This is developped by students for students who wants to test a new way to learn english.
The rules is simple: you have to answer a set of questions about general culture, grammar, listening comprehension, vocabulary... If you answer wrong 3 times, you lose and have to play again since the beginning of the quiz.
At the end of a quiz session, you have the possibility to share your score with others and measure your evolution.
In order to not be bored about the application, users have to possibility to suggest questions.

\setcounter{page}{1}
	\clearpage


\chapter{User Documentation}
	Welcome to \appName.
	\appName is a quiz application about the English culture.\\
	Now we will present you how to use \appName.
	for the next of the presentation of the application, we use this plan:
	\begin{itemize}
		\item Navigation menu
		\item Home page
		\item How to play the game
		\item Best score
		\item Send a question request\\
	\end{itemize}

	\section{Navigation menu}
	The navigation menu was placed on the top of each page of the web site and this offer for the player five Buttons to navigate into the different screens(Figure 1).
	
	\begin{center}
	\includegraphics[width=1\textwidth]{CNave.png}
	(Figure 1)
	\end{center}
	\begin{itemize}
		\item The Game: This Button redirects you to the home page of the application.
		\item Play now!: This Button redirects you to the game page of the application.
		\item High scores: This Button shows you high scores of players.
		\item Suggest a question: This Button redirects you to a form where you can imagine a question for the game.
		\item Logo Button : This button has the same action as the Button ‘The game’ but has a special design randomly set when you load the page.
	\end{itemize}

	\section{Home page}
	When the user log on \appName, he arrives on the Home page screen.
(Figure 2)

\begin{center}
	\includegraphics[width=1\textwidth]{CHome.png}
	(Figure 2)
\end{center}


The Home screen has various sections.
\begin{itemize}
	\item The 'Start game' section. This section allows the user to start a game
	\item The 'Information' section. This section presents rules of the game and the story of \octopusName.
\end{itemize}


	\section{How to play the game}

	To play to \appName game you have to access to the game page via the Button ‘plays now!’ in the navigation menu or via the play button on the home page.

When you are on the play page you have to indicate a username.
This userName will be used for score displaying.
When your username is ready click on “create” button to start the game.(Figure 3)

\begin{center}
	\includegraphics[width=1\textwidth]{CCreate.png}
	(Figure 3)
\end{center}

When you click on “create” button the game start and a loading screen appear to lunch the game.


When the game is lunched the screen is subdivide in Game Section and Information Section.(Figure 4)

\begin{center}
	\includegraphics[width=0.8\textwidth]{CGame.png}
	(Figure 4)
\end{center}

\subsection{Game Section}
 The question of Octopus-Challenge has the next setup:
 	\begin{itemize}
		\item A visual or a YouTube sequence video to have a visual or audio support
		\item The question
		\item Four answer proposition
	\end{itemize}

\subsection{Information Section}
	During the game you can see in the information section various information about the game party:
	\begin{itemize}
	\item How many questions you have answers and how you have to answer to end.
	\item How many lives you have now. At the beginning of the game you have three lives and no one was given during the game.
	\item Timers to see how many time have paste.
	\end{itemize}

now to play the game you just have to click on the correct answer.
If you are right, you go to the next question.
If you are wrong, you lose one of your three lives and an animation appear to indicate you chose the wrong answer, and you have to choose another proposition for the question.
The Game continues until you answer all questions or your life drop to zero.

When you finish, the end screen appears.

\subsection{End Game}
	When you finish the game, you are redirected to the End screen(Figure 5)

\begin{center}
	\includegraphics[width=0.7\textwidth]{CEnd.png}
	(Figure 5)
\end{center}

this screen has:
\begin{itemize}
\item Your Score and your Time
\item The top player scores
\item Button to replay a party
\item Button to share your score and be in the top 10 players if you have a good score
\end{itemize}

\section{Best score}
To see the best score you can go to the high score page with the Best Score button on the navigation menu or to finish a party but limited to the 10 best score.(Figure 6)

\begin{center}
	\includegraphics[width=0.9\textwidth]{bestScore.png}\\
	(Figure 6)
\end{center}

When you go to the best score page you can see the best score, and who make it, how many points he got and how many time he makes to answer all questions.

\section{Send a question request}
If you want to add a question into the game, you can send a question request.
To do this go to the question request page with the button suggest a question in the navigation menu (on the right).(Figure 7)

\begin{center}
	\includegraphics[width=0.9\textwidth]{CQuestion.png}\\
	(Figure 7)
\end{center}


On this page you have a form to indicate all information about question, visual, answer and a real-time render of the question.

When you satisfied with your question you just to click on the submit button to send your question.
Warning: to submit your question you have to select the correct answer via the checkbox on the right of the answer.

\newpage
\chapter{Description technique}

	Ce chapitre présentera en détails les contraintes de ce projet, la manière dont il a été développé ainsi que les problèmes que nous avons dues faire face
	durant les semaines de développement.

	\section{Contraintes}

	Nous avons dégagé deux types de contraintes lors de la phase d'analyse de ce projet:
	\begin{itemize}
		\item Les contraintes liés au sujet (temps de développement, attentes du client).
		\item Les contraintes liés au type de l'outil informatique à développer.\\
	\end{itemize}
Nous avons mis beaucoup de temps à choisir l'outil informatique que nous souhaitions développer. Notre premier choix s'est porté sur un clone du projet Voltaire adapté à la langue de shakespeare.
Cependant, ce choix ne s'est pas révélé motivant. Le choix suivant s'est alors porté sur un quiz à base d'image où l'objectif devait être de trouver le mot correspondant au mot affiché. De plus, ce choix
ne nous semblait pas assez solide pour en faire un outil informatique à part entière. De plus, Ce moyen d'apprendre la langue n'est pas forcément efficace.
Notre choix final s'est porté sur un mélange de tous les choix précédents en y apportant quelque chose d'éducatif et simple à utiliser.\\

Le choix du sujet nous a fait perdre du temps pour la conception et le développement de l'outil. Le développement devait donc se faire sur une plus courte période.
Le développement d'une application web s'est alors présenté à nous. Nous disposons tout les deux des compétences pour travailler dans le web.
De plus, Le web est accessible depuis n'importe quel appareil disposant d'un navigateur internet, point non négligeable pour le succès de l'application.\\

Le sujet étant fixé, d'autres contraintes se sont rajoutés. La première est le fait que l'application de doit être accessible depuis n'importe quel appareil. En effet,
une application ne fonctionne pas et ne s'affiche pas de la même manière sur un ordinateur ou sur un smartphone (écran plus petit, limitations matérielles...). Il sera donc nécéssaire
d'adapter certains aspects de l'application pour la rendre fonctionnelle. La dernière contraintes est l'expérience de l'utilisateur. Si notre application à pour ambition d'améliorer
la compréhension de l'Anglais, la manière dont les informations sont transmises de l'application à l'utilisateur est un point crucial. L'information doit être facilement accessible (peu de clicks, accès rapide, application réactive).
Il saura donc important d'adapter l'application afin dé répondre à cette contrainte dans l'objectif d'apporter la meilleur expérience pour l'utilisateur.


\clearpage
\section{Développement de l'application}


L'application \appName est composé deux parties:
\begin{itemize}
	\item le backend, programme étant exécuté sur le serveur, ayant pour but de répondre aux requètes des clients.
	\item le frontend, programme étant exécuté sur le navigateur web, coeur du projet, permettant de gérer l'affichage des informations sur le navigateur.
\end{itemize}

\section{Identité graphique}
Pour \appName, nous avons voulu adopter un style graphique qui permettrait aux différents éléments de se détacher de l’aspect flat design général utilisé sur l’application.

L’idée retenue parmi tous les styles testés fut des dessins à l’aspect graphique crayonné ou « Draft ».
Nous avons gardé ce style graphique pour plusieurs raisons.
– la mascotte du site avait un rendu très intéressant avec ce type de dessin. 
– Ensuite, le rendu permettait de fournir une image qui se détachait vraiment de l’aspect flat design.
– Une facilité de production via une tablette tactile des différents dessins.
– pouvoir générer des animations sans pour autant beaucoup travailler la qualité des dessins.

\subsection*{la mascotte}
Pour l’application \appName, nous voulions intégrer une mascotte au travers laquelle le joueur pourrait s’identifier voir se confronter.\
Pour cela, nous avions dès le début plusieurs idées de conception, t-elle qu’essayer de la stéréotyper de manière a se qu’elle est un air très british, mais aussi qu’une aptitude sympathique en ressorte.
Nous sommes donc passées par plusieurs types de mascottes telles qu’un aspect humanoïde, des formes géométriques et enfin la pieuvre.
Une fois la conception de base réalisée, nous avons eu l’idée de vouloir lui fournir des expressions en fonction des actions du joueur.
Ainsi nous avons imaginé plusieurs expressions pour \octopusName, qui transite par l’expression faciale, mais aussi par les couleurs utilisées pour le crayonner, de manière à appuyer les sentiments à transmettre. C’est à ce moment que nous avons eu une nouvelle idée pour apporter de la profondeur au sentiment que l’on voulait transmettre, on a alors décidé d’animer les dessins pour former de petit GIF.

\subsection*{Le Logo}
Pour le logo du site, nous avons dans un premier temps réalisé un logo, dans lequel les éléments distinctifs de la mascotte étaient mis en avant.
Ensuite, nous avons eu l’idée de promouvoir les pays anglophones au travers de celui-ci en fesse ressortir les différents drapeaux .


\chapter{Conclusion}
\appName \ was a very interesting experience for the learning of new development technology and for the creativity that we have developed in the application.But the development was not easy and we have a lot of difficult to find an idea for the application. 
Now the application is not finished yet there are a lot of features that can be implemented like, a login setup, multiplayer game, better view on smartphones ... ect.

\chapter{annexe}
200\% Toeic 2017 For many questions

\end{document}
